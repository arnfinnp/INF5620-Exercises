\section{exercise b}

Picard iteration is used for making non-linear PDE's behave like linear PDE's. In our case this is done by using $u$ from the last time step in $\alpha(u)$. 
This means that we now have a known expression for $\alpha(u)$. With Picard iteration eq.(3) becomes

\begin{equation}
  a(u,v) = (\rho(u,v) + \Delta t (\alpha(u_1) \nabla u,\nabla v))dx
\end{equation}

and eq.(4) is unchanged

\begin{equation}
 L(v) = (\rho(I(\vec{x}),v)  + \Delta t(f^n,v))dx
\end{equation}

\lstset{
language=Python,
basicstyle=\footnotesize,
backgroundcolor=\color{grey}
}


The Picard iterations 

\begin{lstlisting}
  u = Function(V)
  eps = 1.0
  tol = 1.0E-5
  iter = 0
  maxiter = 25
  
  while eps > tol and iter < maxiter:
      iter += 1
      solve(a == L, u, bcs)
      diff = u.vector().array() - u_k.vector().array()
      eps = numpy.linalg.norm(diff, ord=numpy.Inf)
      u_k.assign(u)
\end{lstlisting}